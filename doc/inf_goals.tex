You can ask the probability of an atom using \verb|pita| using the predicate using 
\begin{verbatim}
prob(:Query:atom,-Probability:float)|
\end{verbatim}
as in
\begin{verbatim}
?- prob(heads(coin),P).
\end{verbatim}
If the query is non-ground, \verb|prob/2| returns in backtracking the succesful instantiations together with their probability.

When using \verb|mcintyre|, the predicate for querying is 
\begin{verbatim}
mc_prob(:Query:atom,-Probability:float)
\end{verbatim} as in
\begin{verbatim}
?- mc_prob(heads(coin),P).
\end{verbatim}
With \verb|mcintyre|, you can also take a given number of sample with
\begin{verbatim}
mc_sample(:Query:atom,+Samples:int,-Successes:int,-Failures:int,
  -Probability:float).
\end{verbatim}
as in (\href{http://cplint.lamping.unife.it/example/inference/coinmc.pl}{\texttt{coinmc.pl}})
\begin{verbatim}
?- mc_sample(heads(coin),1000,S,F,P).
\end{verbatim}
that samples \verb|heads(coin)| 1000 times and returns in \verb|S| the number of successes, in \verb|F| the number of failures and in \verb|P| the
estimated probability (\verb|S/1000|).

Moreover, you can sample arguments of queries with
\begin{verbatim}
mc_sample_arg(:Query:atom,+Samples:int,?Arg:var,-Values:list).
\end{verbatim}
The predicate samples \verb|Query| a number of \verb|Samples| times. 
\verb|Arg| should be a variable in \verb|Query|.
The predicate returns in \verb|Values| a list of couples \verb|L-N| where
\verb|L| is the list of values of \verb|Arg| for which \verb|Query|
succeeds in a world sampled at random and \verb|N|
is the number of samples.
If \verb|L| is the empty list, it means that for that
sample the query failed. 
If \verb|L| is a list with a 
single element, it means that for that sample the query is 
determinate. 
If, in all couples \verb|L-N|, \verb|L| 
is a list with a 
single element, it means that the clauses in the program 
are mutually exclusive, i.e., that in every sample, only
one clause for each subgoal has the body true. This is one
of the assumptions taken for programs of the PRISM system \cite{DBLP:journals/jair/SatoK01}.
For example
\href{http://cplint.lamping.unife.it/example/inference/pfcglr.pl}{\texttt{pfcglr.pl}} and \href{http://cplint.lamping.unife.it/example/inference/plcg.pl}{\texttt{plcg.pl}} satisfy this constraint while
 \href{http://cplint.lamping.unife.it/example/inference/markov_chain.pl}{\texttt{markov\_chain.pl}} and \href{http://cplint.lamping.unife.it/example/inference/var_obj.pl}{\texttt{var\_obj.pl}} don't.


An example of use of the above predicate is
\begin{verbatim}
?- mc_sample_arg(reach(s0,0,S),50,S,Values). 
\end{verbatim}
of \href{http://cplint.lamping.unife.it/example/inference/markov_chain.pl}{\texttt{markov\_chain.pl}}
that takes 50 samples of \verb|L| in \verb|findall(S,(reach(s0,0,S),L)|.

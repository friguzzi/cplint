\subsection{Graphing the Results}
\label{condqcont}

in \texttt{cplint} on SWISH you can draw graphs
for visualizing the results. There are two types 
of graphs: those that represent individual probability values with a bar chart and those that 
visualize the results of sampling arguments.

You can draw the probability of a query being true and 
being false as a bar chart with \verb|prob_bar(:Query:atom,-Probability:dict)| as in
\begin{verbatim}
?- prob_bar(heads(coin),P).
\end{verbatim}
if you include
\begin{verbatim}
:- use_rendering(c3).
\end{verbatim}
before \verb|:- pita.| \verb|P| will be instantiated with a
dict for rendering with \verb|c3|. It will be shown as a bar chart with
a bar for the probability of \verb|heads(coin)| true and a bar for the probability of \verb|heads(coin)| false.


When using \verb|mcintyre|, you can use
\begin{verbatim}
mc_prob_bar(:Query:atom,-Probability:dict).
\end{verbatim}
as in
\begin{verbatim}
?- mc_prob_bar(heads(coin),P).
\end{verbatim}
to obtain a chart representation of the probability.

You can obtain a bar chart of the samples with
\begin{verbatim}
?- mc_sample_bar(heads(coin),1000,Chart).
\end{verbatim}
that returns in \verb|Chart| a diagram with one bar for the number of successes and 
one bar for the number of failures.

For visualizing the results of sampling arguments you can use
\begin{verbatim}
mc_sample_arg_bar(:Query:atom,+Samples:int,?Arg:var,-Chart:dict).
mc_sample_arg_first_bar(:Query:atom,+Samples:int,
  ?Arg:var,-Chart:dict).
mc_rejection_sample_arg_bar(:Query:atom,:Evidence:atom,+Samples:int,
  ?Arg:var,-Chart:dict).
mc_mh_sample_arg_bar(:Query:atom,:Evidence:atom,+Samples:int,
  +Lag:int,?Arg:var,-Chart:dict).
\end{verbatim} 
that return in \verb|Chart| a bar chart with a bar for each possible sampled value whose size is the number of samples
returning that value.

An example is
\begin{verbatim}
?- mc_sample_arg_bar(reach(s0,0,S),50,S,Chart). 
\end{verbatim}
of \href{http://cplint.lamping.unife.it/example/inference/markov_chain.pl}{\texttt{markov\_chain.pl}}.

Drawing a graph is particularly interesting when
sampling values for continuous arguments of goals.
In this case, you can use the samples to draw the
probability density function of the argument.
The predicate
\begin{verbatim}
histogram(+List:list,+NBins:int,-Chart:dict) 
\end{verbatim}
draws a histogram of the samples in \verb|List| dividing the domain in
 \verb|NBins| bins.
\verb|List| must be a list of couples of the form \verb|[V]-W|
where \verb|V| is a sampled value and \verb|W| is its weight. This is the format of the list of samples returned by argument sampling predicates
except \verb|mc_lw_sample_arg/5| that returns a list of couples \verb|V-W|. 
In this case you can
use
\begin{verbatim}
densities(+PriorList:list,+PostList:list,+NBins:int,-Chart:dict)
\end{verbatim}
that draws a line chart of the density of two sets of samples, usually
 prior and post observations. The samples from the prior are in \verb|PriorList|
as couples \verb|[V]-W|, while the samples from the posterior are in \verb|PostList|
as couples \verb|V-W| where \verb|V| is a value and \verb|W| its weigth.
 The lines are drawn dividing the domain in
 \verb|NBins| bins.
 
For example
\begin{verbatim}
?-  mc_sample_arg(value(0,X),1000,X,L0),
    histogram(L0,40,Chart).
\end{verbatim}
from \href{http://cplint.lamping.unife.it/example/inference/gauss_mean_est.pl}{\texttt{gauss\_mean\_est.pl}}
takes 1000 samples of argument \verb|X| of \verb|value(0,X)| and draws the density of the samples using an histogram.

Instead
\begin{verbatim}
?- mc_sample_arg(value(0,Y),1000,Y,L0),
   mc_lw_sample_arg(value(0,X),
    (value(1,9),value(2,8)),1000,X,L),
   densities(L0,L,NBins,Chart).
\end{verbatim}
from \href{http://cplint.lamping.unife.it/example/inference/gauss_mean_est.pl}{\texttt{gauss\_mean\_est.pl}}
takes 1000 samples of argument \verb|X| of \verb|value(0,X)| before and after observing
\verb|(value(1,9),value(2,8)| and draws the prior and posterior densities of the samples using a line chart.

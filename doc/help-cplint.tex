\documentclass[a4paper,10pt]{scrartcl}
\RequirePackage[hyperindex]{hyperref}

\begin{document}
\title{\texttt{cplint} on SWISH Manual}
\maketitle

%\section{Table of Contents}
%
%\begin{itemize}
%	\item \hyperref[syn]{Syntax} 
%	\item Inference \ref{inf}
%	\item Learning \ref{learning}
%\end{itemize}

\section{Syntax}
\label{syn}

LPAD and CP-logic programs consist of a set of annotated disjunctive clauses.
Disjunction in the head is represented with a semicolon and atoms in the head are separated from probabilities by a colon. For the rest, the usual syntax of Prolog is used.
A general CP-logic clause has the form
\begin{verbatim}
h1:p1 ; ... ; hn:pn :- b1,...,bm,\+ c1,....,\+ cl
\end{verbatim}
 No parentheses are necessary. The \texttt{pi} are numeric expressions. It is up to the user to ensure that the numeric expressions are legal, i.e. that they sum up to less than one.

If the clause has an empty body, it can be represented like this
\begin{verbatim}
h1:p1 ; ... ; hn:pn.
\end{verbatim}
If the clause has a single head with probability 1, the annotation can be omitted and the clause takes the form of a normal prolog clause, i.e. 
\begin{verbatim}
h1 :- b1,...,bm,\+ c1,...,\+ cl.
\end{verbatim}
stands for 
\begin{verbatim}
h1:1 :- b1,...,bm,\+ c1,...,\+ cl.
\end{verbatim}
The coin example of  \cite{VenVer04-ICLP04-IC} is represented as (file \href{http://cplint.lamping.unife.it/example/coin.cpl}{\texttt{coin.cpl}})
\begin{verbatim}
heads(Coin):1/2 ; tails(Coin):1/2 :- 
  toss(Coin),\+biased(Coin).

heads(Coin):0.6 ; tails(Coin):0.4 :- 
  toss(Coin),biased(Coin).

fair(Coin):0.9 ; biased(Coin):0.1.

toss(coin).
\end{verbatim}
The first clause states that if we toss a coin that is not biased it has equal probability of landing heads and tails. The second states that if the coin is biased it has a slightly higher probability of landing heads. The third states that the coin is fair with probability 0.9 and biased with probability 0.1 and the last clause states that we toss a coin with certainty.

Moreover, the bodies of rules may contain the built-in predicates:
\begin{verbatim}
is/2, >/2, </2, >=/2 ,=</2,
=:=/2, =\=/2, true/0, false/0,
=/2, ==/2, \=/2 ,\==/2, length/2
\end{verbatim}
The bodies may also contain the following
 library predicates:
\begin{verbatim}
member/2, max_list/2, min_list/2
nth0/3, nth/3, dif/2, select/3
\end{verbatim}
plus the predicate
\begin{verbatim}
average/2
\end{verbatim}
that, given a list of numbers, computes its arithmetic mean.

The body of rules may also contain the predicate \verb|prob/2| that computes the
probability of an atom, thus allowing nested probability computations.
For example
\begin{verbatim}
a:0.2:-
  prob(b,P),
  P>0.2.
\end{verbatim}
is a valid rule.
\section{Semantics}
\label{semantics}

The semantics of LPADs for the case of programs without functions symbols can be given
as follows. An LPAD defines a probability distribution over normal logic programs called
\emph{worlds}. A world is obtained from an LPAD by first grounding it, by 
selecting a single head atom for each ground clause and by including in the world
the clause with the selected head atom and the body.
The probability of a world is the product of the probabilities associated to the 
heads selected.
The probability of a ground atom (the query) is given by the sum of the probabilities
of the worlds where the query is true.

If the LPAD contains function symbols, the definition is more complex, see
\cite{DBLP:journals/ai/Poole97,DBLP:journals/jair/SatoK01,Rig15-PLP-IW}.

For the semantics of programs with continuous random variables, see \cite{Nitti2016} that defines distributional clauses.
\verb|cplint| allows more freedom in the use of continuous random variables
in expressions, for example
\href{http://cplint.lamping.unife.it/example/inference/kalman_filter.pl}{\texttt{kalman\_filter.pl}} would not be allowed by distributional clauses.

\section{Inference}
\label{inf}
\texttt{cplint} answers queries using the module \verb|pita| or \verb|mcintyre|. The first performs the program transformation technique of \cite{RigSwi10-ICLP10-IC}. Differently from that work, techniques alternative to tabling and answer subsumption are used. The latter performs approximate inference by sampling
using a different program transformation technique and is described in \cite{Rig13-FI-IJ}.

For answering queries, you have to prepare a Prolog file where you first load the inference module (for example \verb|pita|), initialize it with a directive (for example \verb|:- pita|) and then enclose the LPAD
clauses in \verb|:-begin_lpad.| and \verb|:-end_lpad.| For example, the coin program above can be stored in \href{http://cplint.lamping.unife.it/example/inference/coin.pl}{\texttt{coin.pl}} for performing inference
with \verb|pita| as follows
\begin{verbatim}
:- use_module(library(pita)).
:- pita.
:- begin_lpad.
heads(Coin):1/2 ; tails(Coin):1/2:- 
toss(Coin),\+biased(Coin).

heads(Coin):0.6 ; tails(Coin):0.4:- 
toss(Coin),biased(Coin).

fair(Coin):0.9 ; biased(Coin):0.1.

toss(coin).
:- end_lpad.
\end{verbatim}
The same program for \verb|mcintyre| is
\begin{verbatim}
:- use_module(library(mcintyre)).
:- mc.
:- begin_lpad.
heads(Coin):1/2 ; tails(Coin):1/2:- 
toss(Coin),\+biased(Coin).

heads(Coin):0.6 ; tails(Coin):0.4:- 
toss(Coin),biased(Coin).

fair(Coin):0.9 ; biased(Coin):0.1.

toss(coin).
:- end_lpad.
\end{verbatim}
You can have also (non-probabilistic) clauses outside \verb|:-begin/end_lpad.| These are considered as database clauses.
In \verb|pita| subgoals in the body of probabilistic clauses can query them by enclosing the query in \verb|db/1|.
For example (\href{http://cplint.lamping.unife.it/example/inference/testdb.pl}{\texttt{testdb.pl}})
\begin{verbatim}
:- use_module(library(pita)).
:- pita.
:- begin_lpad.
sampled_male(X):0.5:-
  db(male(X)).
:- end_lpad.
male(john).
male(david).
\end{verbatim}
You can also use \verb|findall/3| on subgoals defined by database clauses
(\href{http://cplint.lamping.unife.it/example/inference/persons.pl}{\texttt{persons.pl}})
\begin{verbatim}
:- use_module(library(pita)).
:- pita.
:- begin_lpad.
male:M/P; female:F/P:-
  findall(Male,male(Male),LM),
  findall(Female,female(Female),LF),
  length(LM,M),
  length(LF,F),
  P is F+M.
:- end_lpad.
male(john).
male(david).
female(anna).
female(elen).
female(cathy).
\end{verbatim}
Aggregate predicates on probabilistic subgoals are not implemented due to their high
computational cost (if the aggregation is over $n$ atoms, the values of the
aggregation are potentially $2^n$). The Yap version of \verb|cplint| includes
reasoning algorithms that allows aggregate predicates on probabilistic subgoals,
see \url{http://ds.ing.unife.it/~friguzzi/software/cplint/manual.html}.

In \verb|mcintyre| you can query database clauses in the body of probabilistic clauses without any special syntax. You can also 
use \verb|findall/3|.

\verb|cplint| on SWISH allows two types of programs: LPAD and Prolog. In the first, you create a new LPAD in the editor, in the latter a Prolog program. 

In LPADs, you write in the editor the program without the library import and the compiler directives. \verb|pita| is used for inference.

You ask queries by writing the atom of which you want to compute the probability. In the coin example, the probability of \verb|heads(coin)| can be obtained with the 
\begin{verbatim}
?- heads(coin).
\end{verbatim}

In Prolog programs, you have to enter the program above (\verb|coin.pl|).

\subsection{Unconditional Queries}
\label{uncondq}
You can ask the unconditional probability of an atom using \verb|pita| with the predicate
\begin{verbatim}
prob(:Query:atom,-Probability:float)|
\end{verbatim}
as in
\begin{verbatim}
?- prob(heads(coin),P).
\end{verbatim}
If the query is non-ground, \verb|prob/2| returns in backtracking the succesful instantiations together with their probability.

When using \verb|mcintyre|, the predicate for querying is 
\begin{verbatim}
mc_prob(:Query:atom,-Probability:float)
\end{verbatim} as in
\begin{verbatim}
?- mc_prob(heads(coin),P).
\end{verbatim}
With \verb|mcintyre|, you can also take a given number of sample with
\begin{verbatim}
mc_sample(:Query:atom,+Samples:int,-Successes:int,-Failures:int,
  -Probability:float).
\end{verbatim}
as in (\href{http://cplint.lamping.unife.it/example/inference/coinmc.pl}{\texttt{coinmc.pl}})
\begin{verbatim}
?- mc_sample(heads(coin),1000,S,F,P).
\end{verbatim}
that samples \verb|heads(coin)| 1000 times and returns in \verb|S| the number of successes, in \verb|F| the number of failures and in \verb|P| the
estimated probability (\verb|S/1000|).

If you are just interested in the probability, you can use
\begin{verbatim}
mc_sample(:Query:atom,+Samples:int,-Probability:float) 
\end{verbatim}
as in (\href{http://cplint.lamping.unife.it/example/inference/coinmc.pl}{\texttt{coinmc.pl}})
\begin{verbatim}
?- mc_sample(heads(coin),1000,Prob).
\end{verbatim}
that samples \verb|heads(coin)| 1000 times and returns the
estimated probability that a sample is true (i.e., that a sample succeeds).


Moreover, you can sample arguments of queries with
\begin{verbatim}
mc_sample_arg(:Query:atom,+Samples:int,?Arg:var,-Values:list).
\end{verbatim}
The predicate samples \verb|Query| a number of \verb|Samples| times. 
\verb|Arg| should be a variable in \verb|Query|.
The predicate returns in \verb|Values| a list of couples \verb|L-N| where
\verb|L| is the list of values of \verb|Arg| for which \verb|Query|
succeeds in a world sampled at random and \verb|N|
is the number of samples returning that list of values.
If \verb|L| is the empty list, it means that for that
sample the query failed. 
If \verb|L| is a list with a 
single element, it means that for that sample the query is 
determinate. 
If, in all couples \verb|L-N|, \verb|L| 
is a list with a 
single element, it means that the clauses in the program 
are mutually exclusive, i.e., that in every sample, only
one clause for each subgoal has the body true. This is one
of the assumptions taken for programs of the PRISM system \cite{DBLP:journals/jair/SatoK01}.
For example
\href{http://cplint.lamping.unife.it/example/inference/pfcglr.pl}{\texttt{pfcglr.pl}} and \href{http://cplint.lamping.unife.it/example/inference/plcg.pl}{\texttt{plcg.pl}} satisfy this constraint while
 \href{http://cplint.lamping.unife.it/example/inference/markov_chain.pl}{\texttt{markov\_chain.pl}} and \href{http://cplint.lamping.unife.it/example/inference/var_obj.pl}{\texttt{var\_obj.pl}} don't.


An example of use of the above predicate is
\begin{verbatim}
?- mc_sample_arg(reach(s0,0,S),50,S,Values). 
\end{verbatim}
of \href{http://cplint.lamping.unife.it/example/inference/markov_chain.pl}{\texttt{markov\_chain.pl}}
that takes 50 samples of \verb|L| in \verb|findall(S,(reach(s0,0,S),L)|.

You can sample arguments of queries also with
\begin{verbatim}
mc_sample_arg_first(:Query:atom,+Samples:int,?Arg:var,-Values:list).
\end{verbatim}
samples \verb|Query| a number of \verb|Samples| times 
and returns in \verb|Values| a list of couples \verb|V-N| where 
\verb|V| is the value of \verb|Arg| returned as the first answer by \verb|Query| in 
a world sampled at random and \verb|N| is the number of samples
returning that value.
\verb|V| is failure if the query fails.
\verb|mc_sample_arg_first/4| differs from \verb|mc_sample_arg/4| because the first just computes the first
answer of the query for each sampled world.

%Alternatively, you can use
%\begin{verbatim}
%mc_sample_arg_one(:Query:atom,+Samples:int,?Arg:var,-Values:list)
%\end{verbatim}
%that samples \verb|Query| a number of \verb|Samples| times 
%and returns in \verb|Values| a list of couples \verb|V-N| where 
%\verb|V| is a value sampled with uniform probability from those returned 
%by \verb|Query| in a world sampled at random and \verb|N| is the number of samples
%returning that value.
%\verb|V| is failure if the query fails.

Finally, you can compute expectations with 
\begin{verbatim}
mc_expectation(:Query:atom,+N:int,?Arg:var,-Exp:float).
\end{verbatim}
that computes the expected value of \verb|Arg| in \verb|Query| by
sampling.
It takes \verb|N| samples of \verb|Query| and sums up the value of \verb|Arg| for
each sample. The overall sum is divided by \verb|N| to give \verb|Exp|.

An example of use of the above predicate is
\begin{verbatim}
?- mc_expectation(eventually(elect,T),1000,T,E).
\end{verbatim}
of \href{http://cplint.lamping.unife.it/example/inference/pctl_slep.pl}{\texttt{pctl\_slep.pl}}
that returns in \verb|E| the expected value of \verb|T| by taking 1000 samples.

\subsection{Conditional Queries}
\label{condq}
You can ask the conditional probability of an atom given another atom using \verb|pita| with the predicate 
\begin{verbatim}
prob(:Query:atom,:Evidence:atom,-Probability:float)|
\end{verbatim}
as in
\begin{verbatim}
?- prob(heads(coin),biased(coin),P).
\end{verbatim}
If the query/evidence are non-ground, \verb|prob/3| returns in backtracking ground instantiations together with their probability.

When using \verb|mcintyre|, you can ask conditional queries with rejection sampling or with Metropolis-Hastings Markov Chain Monte Carlo.
In rejection sampling, you first query the evidence and, if the query is successful, query the goal in the same sample, otherwise
the sample is discarded.
In Metropolis-Hastings MCMC, \verb|mcintyre| follows the algorithm proposed in \cite{nampally2014adaptive} (the non adaptive version):
after a sample, a number of sampled probabilistic choices are deleted and the others are retained for the next sample.
The sample is accepted with a probability of \verb|min{1,N0/N1}| where \verb|N0| is the number of choices sampled
in the previous sample and \verb|N1| is the number of choices sampled in the current sample.


You can take a given number of sample with rejection sampling using
\begin{verbatim}
mc_rejection_sample(:Query:atom,:Evidence:atom,+Samples:int,
  -Successes:int,-Failures:int,-Probability:float).
\end{verbatim}
as in (\href{http://cplint.lamping.unife.it/example/inference/coinmc.pl}{\texttt{coinmc.pl}})
\begin{verbatim}
?- mc_rejection_sample(heads(coin),biased(coin),1000,S,F,P).
\end{verbatim}
that takes 1000 samples where \verb|biased(coin)| is true and returns in \verb|S| the number of samples where 
\verb|heads(coin)| is true, in \verb|F| the number of samples where \verb|heads(coin)| is false and in \verb|P| the
estimated probability (\verb|S/1000|).

You can take a given number of sample with Metropolis-Hastings MCMC using
\begin{verbatim}
mc_mh_sample(:Query:atom,:Evidence:atom,+Samples:int,+Lag:int,
  -Successes:int,-Failures:int,-Probability:float).
\end{verbatim}
where \verb|Lag| is the number of sampled choices to forget before taking a new sample.
For example (\href{http://cplint.lamping.unife.it/example/inference/arithm.pl}{\texttt{arithm.pl}})
\begin{verbatim}
?- mc_mh_sample(eval(2,4),eval(1,3),10000,1,T,F,P).
\end{verbatim}
that take 10000 accepted samples and returns in \verb|T| the number of samples where 
\verb|eval(2,4)| is true, in \verb|F| the number of samples where \verb|eval(2,4)| is false and in \verb|P| the
estimated probability (\verb|T/10000|).


Moreover, you can sample arguments of queries with rejection sampling and Metropolis-Hastings MCMC using
\begin{verbatim}
mc_rejection_sample_arg(:Query:atom,:Evidence:atom,
  +Samples:int,?Arg:var,-Values:list).
mc_mh_sample_arg(:Query:atom,:Evidence:atom,
  +Samples:int,+Lag:int,?Arg:var,-Values:list).
\end{verbatim}
that return the distribution of values for \verb|Arg| in \verb|Query| in \verb|Samples| of
\verb|Query| given that \verb|Evidence| is true.
The predicate returns in \verb|Values| a list of couples \verb|L-N| where
\verb|L| is the list of values of \verb|Arg| for which \verb|Query|
succeeds in a world sampled at random where \verb|Evidence| is true and \verb|N|
is the number of samples returning that list of values.

An example of use of the above predicates is
\begin{verbatim}
?- mc_mh_sample_arg(eval(2,Y),eval(1,3),1000,1,Y,V).
\end{verbatim}
of \href{http://cplint.lamping.unife.it/example/inference/arithm.pl}{\texttt{arithm.pl}}.

Finally, you can compute expectations with 
\begin{verbatim}
mc_expectation(:Query:atom,+N:int,?Arg:var,-Exp:float).
\end{verbatim}
that computes the expected value of \verb|Arg| in \verb|Query| by
sampling.
It takes \verb|N| samples of \verb|Query| and sums up the value of \verb|Arg| for
each sample. The overall sum is divided by \verb|N| to give \verb|Exp|.

An example of use of the above predicate is
\begin{verbatim}
?- mc_expectation(eventually(elect,T),1000,T,E).
\end{verbatim}
of \href{http://cplint.lamping.unife.it/example/inference/pctl_slep.pl}{\texttt{pctl\_slep.pl}}
that returns in \verb|E| the expected value of \verb|T| by taking 1000 samples.

To compute conditional expectations, use
\begin{verbatim}
mc_mh_expectation(:Query:atom,:Evidence:atom,+N:int,
  +Lag:int,?Arg:var,-Exp:float).
\end{verbatim}
as in
\begin{verbatim}
?- mc_mh_expectation(eval(2,Y),eval(1,3),1000,1,Y,E).
\end{verbatim}
of \href{http://cplint.lamping.unife.it/example/inference/arithm.pl}{\texttt{arithm.pl}}
that computes the expectation of argument \verb|Y| of \verb|eval(2,Y)| given that 
\verb|eval(1,3)| is true by taking 1000 samples using Metropolis-Hastings MCMC.


You can also see the probability of the query being true and 
being false as a bar chart with \verb|prob_bar(:Query:atom,-Probability:dict)| as in
\begin{verbatim}
?- prob_bar(heads(coin),P).
\end{verbatim}
if you include
\begin{verbatim}
:- use_rendering(c3).
\end{verbatim}
before \verb|:- pita.| \verb|P| will be instantiated with a
dict for rendering with \verb|c3|. It will be shown as a bar chart with
a bar for the probability of \verb|heads(coin)| true and a bar for the probability of \verb|heads(coin)| false.


When using \verb|mcintyre|, you can use
\begin{verbatim}
mc_prob_bar(:Query:atom,-Probability:dict).
\end{verbatim}
as in
\begin{verbatim}
?- mc_prob_bar(heads(coin),P).
\end{verbatim}
to obtain a chart representation of the probability.

You can obtain a bar chart of the samples with
\begin{verbatim}
?- mc_sample_bar(heads(coin),1000,Chart).
\end{verbatim}
that returns in \verb|Chart| a diagram with one bar for the number of successes and 
one bar for the number of failures.

You can also graph the results of sampling arguments with
\begin{verbatim}
mc_sample_arg_bar(:Query:atom,+Samples:int,?Arg:var,-Chart:dict).
mc_sample_arg_first_bar(:Query:atom,+Samples:int,?Arg:var,-Chart:dict) .
\end{verbatim} 
that return in \verb|Chart| a bar chart with a bar for each possible sampled value whose size is the number of samples
returning that value.

An example is
\begin{verbatim}
?- mc_sample_arg_bar(reach(s0,0,S),50,S,Chart). 
\end{verbatim}
of \href{http://cplint.lamping.unife.it/example/inference/markov_chain.pl}{\texttt{markov\_chain.pl}}.

\subsection{Parameters}
The module makes use of a number of parameters in order to control its behavior. They can be set with the directive
\begin{verbatim}
:- set_pita(<parameter>,<value>).
\end{verbatim}
inside the couple \texttt{:-cplint.} and \texttt{:-end\_cplint.} 

The current value can be read with
\begin{verbatim}
?- setting_pita(<parameter>,Value).
\end{verbatim}
from the top-level.
The available parameters are:
\begin{itemize}
\item 
	 \verb|epsilon_parsing|: if (1 - the sum of the probabilities of all the head atoms) is larger than 
    \verb|epsilon_parsing|,
		then \texttt{pita} adds the null event to the head. Default value \texttt{0.00001}.
\item \verb|single_var|: determines how non ground clauses are treated: if \texttt{true}, a single random variable is assigned to the whole non ground clause, 
if \texttt{false}, a different random variable is assigned to every grounding of the clause. Default value \texttt{false}.
\item \verb|depth_bound|: if \texttt{true}, the depth of the derivation of the goal is limited to the value of the \texttt{depth} parameter.  Default value \texttt{false}.
\item  \texttt{depth}: maximum depth of derivations when  \verb|depth_bound| is set to \texttt{true}. Default value \texttt{2}.
\end{itemize}



\section{Learning}
\label{learning}
The following learning algorithms are available:
\begin{itemize}
\item EMBLEM (EM over Bdds for probabilistic Logic programs Efficient Mining): an implementation of EM for learning parameters that computes expectations directly on BDDs \cite{BelRig11-IDA}, \cite{BelRig11-CILC11-NC}, \cite{BelRig11-TR}
\item SLIPCOVER (Structure LearnIng of Probabilistic logic programs by searChing OVER the clause space): an algorithm for learning the structure of programs by searching the clause space and the theory space separately \cite{BelRig13-TPLP-IJ}
\end{itemize}

\subsection{Input}
To execute the learning algorithms, prepare a Prolog file divided in five parts
\begin{itemize}
\item preamble
\item  background knowledge, i.e., knowledge valid for all interpretations
\item  LPAD/CPL-program for you which you want to learn the parameters (optional)
\item language bias information
\item  example interpretations 
\end{itemize}
The preamble must come first, the order of the other parts can be changed.

For example, consider the Bongard problems of \cite{RaeLae95-ALT95}. 
%The \texttt{pack/cplint/ prolog/examples/learning} folder in SWI-Prolog home contains some example learning files. 
\href{http://cplint.lamping.unife.it/example/learning/bongard.pl}{\texttt{bongard.pl}} and \href{http://cplint.lamping.unife.it/example/learning/bongardkeys.pl}{\texttt{bongardkeys.pl}} represent a Bongard problem.
Let us consider \href{http://cplint.lamping.unife.it/example/learning/bongard.pl}{\texttt{bongard.pl}}.

\subsubsection{Preamble}
In the preamble, the SLIPCOVER library is loaded with
\begin{verbatim}
:- use_module(library(slipcover)).
\end{verbatim}
%Then, if you are using your file in cplint on SWISH, you could add
%\begin{verbatim}
%:- if(current_predicate(use_rendering/1)).
%:- use_rendering(c3).
%:- use_rendering(lpad).
%:- endif.
%\end{verbatim}
%if you want a nice representation of the output (in particular, if you want graphs of the ROC and PR curves).
Now you can initialize SLIPCOVER with
\begin{verbatim}
:- sc.
\end{verbatim}
At this point you can start setting parameters for SLIPCOVER such as for example
\begin{verbatim}
:- set_sc(megaex_bottom,20).
:- set_sc(max_iter,2).
:- set_sc(max_iter_structure,5).
:- set_sc(verbosity,1).
\end{verbatim}
We will see later the list of available parameters.
A particularly important parameter is \verb|verbosity|: if set
to 1, nothing is printed and learning is  fastest, if set to 3 much information is printed and learning is slowest, 2 is in between.
This ends the preamble.

\subsubsection{Background and Initial LPAD/CPL-program}
%
Now you can specify the background knowledge with a 
fact of the form 
\begin{verbatim}
bg(<list of terms representing clauses>).
\end{verbatim}
where the clauses must currently be deterministic.
Alternatively, you can specify a set of clauses by including them in 
a section between
\verb|:- begin_bg.| and \verb|:- end_bg.| For example
\begin{verbatim}
:- begin_bg.
replaceable(gear).
replaceable(wheel).
replaceable(chain).
not_replaceable(engine).
not_replaceable(control_unit).
component(C):-
  replaceable(C).
component(C):-
  not_replaceable(C).
:- end_bg.
\end{verbatim}
from the \href{http://cplint.lamping.unife.it/example/learning/mach.pl}{\texttt{mach.pl}} example.
If you specify both a \verb|bg/1| fact and a section, the clauses of the two will be combined.


Moreover, you can specify an initial program with a fact of the form 
\begin{verbatim}
in(<list of terms representing clauses>).
\end{verbatim}
The initial program is used in parameter learning for providing 
the structure. The indicated parameters do not matter as they are first randomized.
Remember to enclose each clause in parentheses because \verb|:-| has the highest precedence.

For example, \href{http://cplint.lamping.unife.it/example/learning/bongard.pl}{\texttt{bongard.pl}} has the initial program 
\begin{verbatim}
in([(pos:0.197575 :-
       circle(A),
       in(B,A)),
    (pos:0.000303421 :-
       circle(A),
       triangle(B)), 
    (pos:0.000448807 :-
       triangle(A),
       circle(B))]).
\end{verbatim}
%Both facts should be present. If there are no background/input clauses then write \verb|bg([]).|/\verb|in([]).|
Alternatively, you can specify an input program in a section between \verb|:- begin_in.| and \verb|:- end_in.|, as for example
\begin{verbatim}
:- begin_in.
pos:0.197575 :-
  circle(A),
  in(B,A).
pos:0.000303421 :-
  circle(A),
  triangle(B).
pos:0.000448807 :-
  triangle(A),
  circle(B).
:- end_in.
\end{verbatim}
If you specify both a \verb|in/1| fact and a section, the clauses of the two will be combined.



\subsubsection{Language Bias}
%
The language bias part contains the declarations of the input and output predicates.
Output predicates are declared as
\begin{verbatim}
output(<predicate>/<arity>).
\end{verbatim}
and indicate the predicate whose atoms you want to predict. Derivations for the atoms for this predicates in the input data
are built by the system. These are the predicates for which new clauses are generated.

Input predicates are those whose atoms you are not interested in predicting. You can declare closed world input predicates with
\begin{verbatim}
input_cw(<predicate>/<arity>).
\end{verbatim}
For these predicates, the only true atoms are those in the interpretations and those derivable from them using the background knowledge, the clauses in the input/hypothesized program are not used to derive atoms for these predicates. Moreover,   clauses of the background knowledge that define closed world input predicates and that call an output predicate in the body will not be used for deriving examples.

Open world input predicates are declared with
\begin{verbatim}
input(<predicate>/<arity>).
\end{verbatim}
In this case, if a subgoal for such a predicate is encountered when deriving a subgoal for the output predicates, 
both the facts in the interpretations, those derivable from them and the background knowledge, the background clauses and the clauses of the input program are used.

Then, you have to specify the language bias by means of mode declarations in the style of 
\href{http://www.doc.ic.ac.uk/\string ~shm/progol.html}{Progol}.
\begin{verbatim}
modeh(<recall>,<predicate>(<arg1>,...)).
\end{verbatim}
specifies the atoms that can appear in the head of clauses, while
\begin{verbatim}
modeb(<recall>,<predicate>(<arg1>,...)).
\end{verbatim}
specifies the atoms that can appear in the body of clauses.
\texttt{<recall>} can be an integer or \texttt{*}.
\texttt{<recall>} indicates how many atoms for the predicate specification are
retained in the bottom clause during a saturation step. \texttt{*} stands for all those that are found. Otherwise the indicated number is randomly chosen.

Two specialization modes are available: \verb|bottom| and \verb|mode|.
In the first, a bottom clause is built and the literals to be added during 
refinement are taken from it. In the latter, no bottom clause is built and
the literals to be added during refinement are generated 
directly from the mode declarations. 

Arguments of the form
\begin{verbatim}
+<type>
\end{verbatim}
specifies that the argument should be an input variable of type \texttt{<type>}, i.e., a variable replacing a \texttt{+<type>} argument in the head or a \texttt{-<type>} argument in a preceding literal in the current hypothesized clause.

Another argument form is
\begin{verbatim}
-<type>
\end{verbatim}
for specifying that the argument should be a output variable of type \texttt{<type>}. 
Any variable can replace this argument, either input or output.
The only constraint on output variables is that those in the head of the current hypothesized 
clause must appear as output variables in an atom of the body.

Other forms are
\begin{verbatim}
#<type>
\end{verbatim}
for specifying an argument which should be replaced by a constant of type \texttt{<type>} in the bottom clause but should not be used for replacing input variables of the following literals when building the bottom clause or 
\begin{verbatim}
-#<type>
\end{verbatim}
for specifying an argument which should be replaced by a constant of type \texttt{<type>} in the bottom clause and that should be used for replacing input variables of the following literals when building the bottom clause. 
%\verb|#| and \verb|-#| differ only in the creation of the bottom clause.
\begin{verbatim}
<constant>
\end{verbatim}
for specifying a constant.

Note that arguments of the form
\verb|#<type>| \verb|-#<type>| are not available in 
specialization mode \verb|mode|, if you want constants to appear in 
the literals you have to indicate them one by one in the mode declarations.


An example of language bias for the Bongard domain is
\begin{verbatim}
output(pos/0).

input_cw(triangle/1).
input_cw(square/1).
input_cw(circle/1).
input_cw(in/2).
input_cw(config/2).

modeh(*,pos).
modeb(*,triangle(-obj)).
modeb(*,square(-obj)).
modeb(*,circle(-obj)).
modeb(*,in(+obj,-obj)).
modeb(*,in(-obj,+obj)).
modeb(*,config(+obj,-#dir)).
\end{verbatim}
SLIPCOVER also requires facts for the \verb|determination/2| predicate  Aleph-style that indicate which predicates can appear in the body of clauses. 
For example
\begin{verbatim}
determination(pos/0,triangle/1).
determination(pos/0,square/1).
determination(pos/0,circle/1).
determination(pos/0,in/2).
determination(pos/0,config/2).
\end{verbatim}
state that \verb|triangle/1| can appear in the body of clauses for \verb|pos/0|.

SLIPCOVER also allows mode declarations of the form
\begin{verbatim}
modeh(<r>,[<s1>,...,<sn>],[<a1>,...,<an>],[<P1/Ar1>,...,<Pk/Ark>]). 
\end{verbatim}
These mode declarations are used to generate clauses with more than two head atoms. In them, \verb|<s1>,...,<sn>| are schemas,  \verb|<a1>,...,<an>| are atoms such that \verb|<ai>| is obtained from $\verb|<si>|$ by replacing placemarkers with variables, 
\verb|<Pi/Ari>| are the predicates admitted in the body. \verb|<a1>,...,<an>| are used to indicate which variables should be shared by the atoms in the head.
An example of such a mode declaration (from \texttt{uwcselearn.pl}) is
\begin{verbatim}
modeh(*,
[advisedby(+person,+person),tempadvisedby(+person,+person)],
[advisedby(A,B),tempadvisedby(A,B)],
[professor/1,student/1,hasposition/2,inphase/2,
publication/2,taughtby/3,ta/3,courselevel/2,yearsinprogram/2]).
\end{verbatim}
%
If you want to specify negative literals for addition in the body of clauses,
you should define a new predicate in the background as in
\begin{verbatim}
not_worn(C):-
  component(C),
  \+ worn(C).
one_worn:-
  worn(_).
none_worn:-
  \+ one_worn.
\end{verbatim}
from \href{http://cplint.lamping.unife.it/example/learning/mach.pl}{\texttt{mach.pl}} and add the new predicate in a \verb|modeb/2| fact
\begin{verbatim}
modeb(*,not_worn(-comp)).
modeb(*,none_worn).
\end{verbatim}
Note that successful negative literals do not instantiate the variables, so if you want
a variable appearing in a negative literal to be an output variable you must instantiate 
before calling the negative literals.
The new predicates must also be declared as input
\begin{verbatim}
input_cw(not_worn/1).
input_cw(none_worn/0).
\end{verbatim}
Lookahead can also be specified with facts of the form
\begin{verbatim}
lookahead(<literal>,<list of literals>).
\end{verbatim}
In this case when a literal matching \verb|<literal>| is added to the body of clause during refinement, then also
the literals matching \verb|<list of literals>| will be added.
An example of such declaration (from \href{http://cplint.lamping.unife.it/example/learning/muta.pl}{\texttt{muta.pl}}) is
\begin{verbatim}
lookahead(logp(_),[(_=_))]).
\end{verbatim}
Note that
\verb|<list of literals>| is copied with \verb|copy_term/2| before matching, so
variables in common between \verb|<literal>| and \verb|<list of literals>|
may not be in common in the refined clause.

It is also possible to specify that a literal can only be added together with 
other literals with facts of the form 
\begin{verbatim}
lookahead_cons(<literal>,<list of literals>).
\end{verbatim}
In this case \verb|<literal>| is added to the body of clause during refinement only together with
literals matching \verb|<list of literals>|.
An example of such declaration is
\begin{verbatim}
lookahead_cons(logp(_),[(_=_))]).
\end{verbatim}
Also here 
\verb|<list of literals>| is copied with \verb|copy_term/2| before matching, so
variables in common between \verb|<literal>| and \verb|<list of literals>|
may not be in common in the refined clause.

Moreover, we can specify lookahead with
\begin{verbatim}
lookahead_cons_var(<literal>,<list of literals>).
\end{verbatim}
In this case \verb|<literal>| is added to the body of clause during refinement only together with
literals matching \verb|<list of literals>| and \verb|<list of literals>| is not copied before matching, so
variables in common between \verb|<literal>| and \verb|<list of literals>|
are in common also in the refined clause. This is allowed only with
\verb|specialization| set to \verb|bottom|.
An example of such declaration is
\begin{verbatim}
lookahead_cons_var(logp(B),[(B=_))]).
\end{verbatim}

\subsubsection{Example Interpretations}
The last part of the file contains the data.
You can specify data with two modalities:
models and keys.
In the models type, you specify an example model (or interpretation or megaexample) as a list of Prolog facts initiated by 
\texttt{begin(model(<name>)).} and terminated by \texttt{end(model(<name>)).} as in
\begin{verbatim}
begin(model(2)).
pos.
triangle(o5).
config(o5,up).
square(o4).
in(o4,o5).
circle(o3).
triangle(o2).
config(o2,up).
in(o2,o3).
triangle(o1).
config(o1,up).
end(model(2)).
\end{verbatim}
The interpretations may contain a fact of the form
\begin{verbatim}
prob(0.3).
\end{verbatim}
assigning a probability (0.3 in this case) to the interpretations. If this is omitted, the probability of each interpretation is considered equal to $1/n$ where $n$ is the total number of interpretations. \verb|prob/1| can be used to set a different multiplicity for the interpretations.

The facts in the interpretation are loaded in SWI-Prolog database by adding an extra initial argument equal to the name of the model.
After each interpretation is loaded, a fact of the form \verb|int(<id>)| is asserted, where \verb|id| is the name of the interpretation. This can be used in
order to retrieve the list of interpretations.

Alternatively, with the keys modality, you can directly write the facts and the first argument will be interpreted as a model identifier. The above interpretation in the keys modality is
\begin{verbatim}
pos(2).
triangle(2,o5).
config(2,o5,up).
square(2,o4).
in(2,o4,o5).
circle(2,o3).
triangle(2,o2).
config(2,o2,up).
in(2,o2,o3).
triangle(2,o1).
config(2,o1,up).
\end{verbatim}
which is contained in the \href{http://cplint.lamping.unife.it/example/learning/bongardkeys.pl}{\texttt{bongardkeys.pl}}
This is also how model \verb|2| above is stored in SWI-Prolog database.
The two modalities, models and keys, can be mixed in the same file.

Note that you can add background knowledge that is not probabilistic directly to the file writing the clauses taking into account the model argument. For example (\texttt{carc.pl})
contains
\begin{verbatim}
connected(_M,Ring1,Ring2):-
  Ring1 \= Ring2,
  member(A,Ring1),
  member(A,Ring2), !.

symbond(Mod,A,B,T):- bond(Mod,A,B,T).
symbond(Mod,A,B,T):- bond(Mod,B,A,T).
\end{verbatim}
where the first argument of all the atoms is the model.

Example \href{http://cplint.lamping.unife.it/example/learning/registration.pl}{\texttt{registration.pl}} contains for example
\begin{verbatim}
party(M,P):-
  participant(M,_, _, P, _).
\end{verbatim}
that defines intensionally the target predicate \verb|party/1|. Here \verb|M| is the model and \verb|participant/4| is defined in the interpretations.
You can also define intensionally the negative examples with
\begin{verbatim}
neg(party(M,yes)):- party(M,no).
neg(party(M,no)):- party(M,yes).
\end{verbatim}
Then you must indicate how the examples are divided in folds with facts of the form:
\verb|fold(<fold_name>,<list of model identifiers>)|, as for example
\begin{verbatim}
fold(train,[2,3,...]).
fold(test,[490,491,...]).
\end{verbatim}
As the input file is a Prolog program, you can define intensionally the folds as in
\begin{verbatim}
fold(all,F):-
  findall(I,int(I),F).
\end{verbatim}
\verb|fold/2| is dynamic so you can also write (\href{http://cplint.lamping.unife.it/example/learning/registration.pl}{\texttt{registration.pl}})
\begin{verbatim}
:- fold(all,F),
   sample(4,F,FTr,FTe),
   assert(fold(rand_train,FTr)),
   assert(fold(rand_test,FTe)).
\end{verbatim}
which however must be inserted after the input interpretations otherwise the facts for \verb|int/1| will not be available and
the fold \verb|all| would be empty. This command uses  \verb|sample(N,List,Sampled,Rest)| exported from \verb|slipcover| that samples \verb|N| elements from \verb|List| and returns the sampled elements in \verb|Sampled| and the rest in \verb|Rest|. If \verb|List| has \verb|N| elements or less, \verb|Sampled| is equal to \verb|List| 
and \verb|Rest| is empty.


\subsection{Commands}
\subsubsection{Parameter Learning}
To execute EMBLEM, prepare an input file in the editor panel as indicated above 
and call
\begin{verbatim}
?- induce_par(<list of folds>,P).
\end{verbatim}
where \verb|<list of folds>| is a list of the folds for training and
\verb|P| will contain the input program with updated parameters.

For example \href{http://cplint.lamping.unife.it/example/bongard.pl}{\texttt{bongard.pl}}, you can 
perform parameter learning on the \verb|train| fold with 
\begin{verbatim}
?- induce_par([train],P).
\end{verbatim}
A program can also be tested on a test set with
\begin{verbatim}
?- test(<program>,<list of folds>,LL,AUCROC,ROC,AUCPR,PR).
\end{verbatim}
where \verb|<program>| is a list of terms representing clauses and
\verb|<list of folds>| is a list of folds.
This returns the log likelihood of the test examples in \verb|LL|, the Area Under the ROC curve in \verb|AUCROC|, a dictionary containing the list of points (in the form of Prolog pairs \verb|x-y|) of the ROC curve in \verb|ROC|,
the Area Under the PR curve in \verb|AUCPR|, a dictionary containing the list of points of the PR curve in \verb|PR|.

For example, to test on fold \verb|test| the program learned on fold \verb|train| you can run the query
\begin{verbatim}
?- induce_par([train],P),
   test(P,[test],LL,AUCROC,ROC,AUCPR,PR).
\end{verbatim}
Or you can test the input program on the fold \verb|test| with
\begin{verbatim}
?- in(P),
   test(P,[test],LL,AUCROC,ROC,AUCPR,PR).
\end{verbatim}
By including
\begin{verbatim}
:- use_rendering(c3).
:- use_rendering(lpad).
\end{verbatim}
in the code before \verb|:- sc.| the curves will be shown as graphs and the output program will be pretty printed.



\subsubsection{Structure Learning}
To execute SLIPCOVER,
prepare an input file in the editor panel as indicated above 
and call
\begin{verbatim}
?- induce(<list of folds>,P).
\end{verbatim}
where \verb|<list of folds>| is a list of the folds for training and
\verb|P| will contain the learned program.

For example \href{http://cplint.lamping.unife.it/example/bongard.pl}{\texttt{bongard.pl}}, you can perform structure learning on the \verb|train| fold with 
\begin{verbatim}
?- induce([train],P).
\end{verbatim}
A program can also be tested on a test set with \verb|test/7| as
described above.

\subsection{Parameters}
Parameters are set with  commands of the form
\begin{verbatim}
:- set_sc(<parameter>,<value>).
\end{verbatim}
The available parameters are:
\begin{itemize}
\item \verb|depth_bound|: (values: \verb|{true,false}|,  default value: \texttt{true}) if \texttt{true}, the depth of the derivation of the goal is limited to the value of the \texttt{depth} parameter. 
\item \verb|depth| (values: integer, default value: 2): depth of derivations if  \verb|depth_bound|  is set to \verb|true|
\item \verb|single_var| (values: \verb|{true,false}|, default value: \verb|false|): if set to \verb|true|, there is a random variable for each clause, instead of a different random variable for each grounding of each clause
\item \verb|epsilon_em| (values: real, default value: 0.1): if the difference in the log likelihood in two successive parameter EM iteration is smaller
than \verb|epsilon_em|, then EM stops 
\item \verb|epsilon_em_fraction| (values: real, default value: 0.01): if the difference in the log likelihood in two successive parameter EM iteration is smaller
than \verb|epsilon_em_fraction|*(-current log likelihood), then EM stops
\item \verb|iter| (values: integer, defualt value: 1): maximum number of iteration of EM parameter learning. If set to -1, no maximum number of iterations is imposed
\item \verb|iterREF| (values: integer, defualt value: 1, valid for  
 SLIPCOVER):
 maximum number of iteration of EM parameter learning for refinements. If set to -1, no maximum number of iterations is imposed.
\item \verb|random_restarts_number| (values: integer, default value: 1, valid for EMBLEM and SLIPCOVER): number of random restarts of parameter EM learning
\item \verb|random_restarts_REFnumber| (values: integer, default value: 1, valid for  SLIPCOVER): number of random restarts of parameter EM learning for refinements
\item \verb|setrand| (values: rand(integer,integer,integer)): seed for the random functions, see SWI-Prolog manual for the allowed values
\item \verb|logzero| (values: negative real, default value $\log(0.000001)$): value assigned to $\log 0$
\item \verb|max_iter| (values: integer, default value: 10, valid for  SLIPCOVER): number of interations of beam search
\item \verb|max_var| (values: integer, default value: 4, valid for 
SLIPCOVER): maximum number of distinct variables in a clause
\item \verb|beamsize|  (values: integer, default value: 100, valid for SLIPCOVER): size of the beam 
\item \verb|megaex_bottom| (values: integer, default value: 1, valid for SLIPCOVER): number of mega-examples on which to build the bottom clauses
\item \verb|initial_clauses_per_megaex| (values: integer, default value: 1, valid for SLIPCOVER): 
 number of bottom clauses to build for each mega-example (or 
 model or interpretation)
\item \verb|d| (values: integer, default value: 1, valid for SLIPCOVER): 
 number of saturation steps when building the bottom clause
\item \verb|max_iter_structure| (values: integer, default value: 10000, valid for SLIPCOVER): 
maximum  number of theory search iterations
\item \verb|background_clauses| (values: integer, default value: 50, valid for SLIPCOVER): 
 maximum numbers of background clauses
\item \verb|maxdepth_var| (values: integer, default value: 2, valid for SLIPCOVER): maximum depth of
variables in clauses (as defined in \cite{DBLP:journals/ai/Cohen95}).
\item \verb|neg_ex| (values:  \verb|given|, \verb|cw|, default value: \verb|cw|): if  set to \verb|given|, the negative examples in testing
are taken from the test folds interpretations, i.e., those examples \verb|ex| stored as \verb|neg(ex)|; if set to \verb|cw|, the negative examples are generated according to the closed world assumption, i.e., all atoms for target predicates that are not positive examples. The set of all atoms is obtained by collecting the set of constants for each type of the arguments of the target predicate.
\item \verb|verbosity| (values: integer in [1,3], default value: 1): level of verbosity of the algorithms
\end{itemize}


\section{Manual in PDF}
A PDF version of the manual is available at
\url{https://github.com/friguzzi/cplint/blob/master/doc/help-cplint.pdf}.
\section{Bibliography}
\bibliographystyle{plain}
\bibliography{bib}
\end{document}
\documentclass[a4paper,10pt]{scrartcl}
\RequirePackage[hyperindex]{hyperref}

\begin{document}
\title{\texttt{cplint} on SWISH Manual}
\maketitle

%\section{Table of Contents}
%
%\begin{itemize}
%	\item \hyperref[syn]{Syntax} 
%	\item Inference \ref{inf}
%	\item Learning \ref{learning}
%\end{itemize}

\section{Syntax}
\label{syn}
\texttt{cplint} permits the definition of discrete probability distributions and continuous probaility
densities.
\subsection{Discrete Probability Distributions}
\label{discrete}
LPAD and CP-logic programs consist of a set of annotated disjunctive clauses.
Disjunction in the head is represented with a semicolon and atoms in the head are separated from probabilities by a colon. For the rest, the usual syntax of Prolog is used.
A general CP-logic clause has the form
\begin{verbatim}
h1:p1 ; ... ; hn:pn :- Body.
\end{verbatim}
where \verb|Body| is a conjunction of goals as in Prolog.
 No parentheses are necessary. The \texttt{pi} are numeric expressions. It is up to the user to ensure that the numeric expressions are legal, i.e. that they sum up to less than one.

If the clause has an empty body, it can be represented like this
\begin{verbatim}
h1:p1 ; ... ; hn:pn.
\end{verbatim}
If the clause has a single head with probability 1, the annotation can be omitted and the clause takes the form of a normal prolog clause, i.e. 
\begin{verbatim}
h1 :- Body.
\end{verbatim}
stands for 
\begin{verbatim}
h1:1 :- Body.
\end{verbatim}
The coin example of  \cite{VenVer04-ICLP04-IC} is represented as (file \href{http://cplint.lamping.unife.it/example/inference/coin.cpl}{\texttt{coin.cpl}})
\begin{verbatim}
heads(Coin):1/2 ; tails(Coin):1/2 :- 
  toss(Coin),\+biased(Coin).

heads(Coin):0.6 ; tails(Coin):0.4 :- 
  toss(Coin),biased(Coin).

fair(Coin):0.9 ; biased(Coin):0.1.

toss(coin).
\end{verbatim}
The first clause states that if we toss a coin that is not biased it has equal probability of landing heads and tails. The second states that if the coin is biased it has a slightly higher probability of landing heads. The third states that the coin is fair with probability 0.9 and biased with probability 0.1 and the last clause states that we toss a coin with certainty.

Moreover, the bodies of rules may contain built-in predicates, predicates
from the libraries \verb|lists|, \verb|apply| and \verb|clpr/nf_r|
plus the predicate
\begin{verbatim}
average/2
\end{verbatim}
that, given a list of numbers, computes its arithmetic mean.

The body of rules may also contain the predicate \verb|prob/2| that computes the
probability of an atom, thus allowing nested probability computations.
For example (\href{http://cplint.lamping.unife.it/example/inference/meta.pl}{\texttt{meta.pl}})
\begin{verbatim}
a:0.2:-
  prob(b,P),
  P>0.2.
\end{verbatim}
is a valid rule.

Moreover, the probabilistic annotations can be variables, as in 
(\href{http://cplint.lamping.unife.it/example/inference/flexprob.pl}{\texttt{flexprob.pl}}))
\begin{verbatim}
red(Prob):Prob.

draw_red(R, G):-
  Prob is R/(R + G),
  red(Prob).
\end{verbatim}
Variables in probabilistic annotations must be ground when resolution reaches the end of the body, 
otherwise an exception is raised.

Alternative ways of specifying probability distribution include
\begin{verbatim}
A:discrete(Var,D):-Body.
\end{verbatim}
or
\begin{verbatim}
A:finite(Var,D):-Body.
\end{verbatim}
where \verb|A| is an atom containg variable \verb|Var| and \verb|D|
is a list of couples \verb|Value:Prob| assigning probability \verb|Prob|
to \verb|Value|. 
Moreover, you can use
\begin{verbatim}
A:uniform(Var,D):-Body.
\end{verbatim}
where \verb|A| is an atom containg variable \verb|Var| and \verb|D|
is a list of values each taking the same probability (1 over the length
of \verb|D|).
\subsubsection{ProbLog Syntax}
You can also use ProbLog \cite{DBLP:conf/ijcai/RaedtKT07} syntax, so a general clause takes the form
\begin{verbatim}
p1::h1 ; ... ; pn::hn :- Body
\end{verbatim}
where the \texttt{pi} are numeric expressions. 

\subsubsection{PRISM Syntax}
You can also use PRISM \cite{DBLP:conf/ijcai/SatoK97} syntax, so a program is composed of
a set of regular Prolog rules whose body may contain calls to the \verb|msw/2| predicate (multi-ary 
switch). A call \verb|msw(term,value)| means that a random variable associated to \verb|term|
assumes value \verb|value|. The admissible values  for a discrete random variable are 
specified using facts for the \verb|values/2| predicate of the form
\begin{verbatim}
values(T,L).
\end{verbatim}
where \verb|T| is a term (possibly containing variables) and \verb|L| is a list of values.
The distribution over values is specified using directives for \verb|set_sw/2| of the form
\begin{verbatim}
:- set_sw(T,LP).
\end{verbatim}
where \verb|T| is a term (possibly containing variables) and \verb|LP| is a list of
probability values.
Remember that in PRISM each call to \verb|msw/2| refers to a different random
variable, i.e., no memoing is performed, differently from the case of LPAD/CP-Logic/ProbLog.

For example, the coin example above in PRISM syntax becomes
\begin{verbatim}
values(throw(_),[heads,tails]).
:- set_sw(throw(fair),[0.5,0.5]).
:- set_sw(throw(biased),[0.6,0.4]).
values(fairness,[fair,biased]).
:- set_sw(fairness,[0.9,0.1]).
res(Coin,R):- toss(Coin),fairness(Coin,Fairness),msw(throw(Fairness),R).
fairness(_Coin,Fairness) :- msw(fairness,Fairness).
toss(coin).
\end{verbatim}
\subsection{Continuous Probability Densities}
\label{cont}

\verb|cplint| handles continuous random variables as well with its
sampling inference module.
To specify a probability density on an argument \verb|Var| of an atom
\verb|A| you can used rules of the form
\begin{verbatim}
A:Density:- Body
\end{verbatim}
where \verb|Density| is a special atom identifying a probability  density on variable \verb|Var| and \verb|Body| (optional) is a regular clause body.
Allowed \verb|Density| atoms are
\begin{itemize}
\item \verb|uniform(Var,L,U)|: \verb|Var| is uniformly distributed in $[L,U]$
\item \verb|gaussian(Var,Mean,Variance)|: \verb|Var| follows a Gaussian distribution with mean \verb|Mean| and variance \verb|Variance|
\item \verb|dirichlet(Var,Par)|: \verb|Var| is a list of real
numbers following a Dirichlet distribution with $\alpha$ parameters specified
by the list \verb|Par|
\item \verb|gamma(Var,Shape,Scale)|  \verb|Var| follows a gamma distribution 
with shape parameter \verb|Shape| and scale parameter \verb|Scale|.
\item \verb|beta(Var,Alpha,Beta)|  \verb|Var| follows a beta distribution 
with parameters \verb|Alpha| and \verb|Beta|.
\item \verb|poisson(Var,Lambda)|  \verb|Var| follows a Poisson distribution 
with parameter \verb|Lambda|.
\end{itemize}
For example
\begin{verbatim}
g(X): gaussian(X,0, 1).
\end{verbatim}
states that argument \verb|X| of \verb|g(X)| follows a Gaussian 
distribution with mean 0 and variance 1.

For example, \href{http://cplint.lamping.unife.it/example/inference/gaussian_mixture.pl}{\texttt{gaussian\_mixture.pl}} defines a mixture of two Gaussians:
\begin{verbatim}
heads:0.6;tails:0.4.
g(X): gaussian(X,0, 1).
h(X): gaussian(X,5, 2).
mix(X) :- heads, g(X).
mix(X) :- tails, h(X).
\end{verbatim}
The argument \verb|X| of
\verb|mix(X)| follows a distribution that is a mixture of two Gaussian,
one with mean 0 and variance 1 with probability 0.6 and one with 
mean 5 and variance 2 with probability 0.4.

The parameters of the distribution atoms can be taken from the probabilistic
atom, the example \href{http://cplint.lamping.unife.it/example/inference/gauss_mean_est.pl}{\texttt{gauss\_mean\_est.pl}}
\begin{verbatim}
value(I,X) :-
  mean(M),
  value(I,M,X).
mean(M): gaussian(M,1.0, 5.0).
value(_,M,X): gaussian(X,M, 2.0).
\end{verbatim}
states that for an index \verb|I| the continuous variable \verb|X| is 
sampled from a Gaussian whose variance is 2 and whose mean is sampled from a Guassian with mean 1 and
variance 5.

Any operation is allowed on continuous random variables. For example
\href{http://cplint.lamping.unife.it/example/inference/kalman_filter.pl}{\texttt{kalman\_filter.pl}}
\begin{verbatim}
kf(N,O, T) :-
  init(S),
  kf_part(0, N, S,O,T).
kf_part(I, N, S,[V|RO], T) :-
  I < N,
  NextI is I+1,
  trans(S,I,NextS),
  emit(NextS,I,V),
  kf_part(NextI, N, NextS,RO, T).
kf_part(N, N, S, [],S).
trans(S,I,NextS) :-
  {NextS =:= E + S},
  trans_err(I,E).
emit(NextS,I,V) :-
  {NextS =:= V+X},
  obs_err(I,X).
init(S):gaussian(S,0,1).
trans_err(_,E):gaussian(E,0,2).
obs_err(_,E):gaussian(E,0,1).
\end{verbatim}
encodes a Kalman filter. Continuous random variables are involved
in arithmetic expressions (in \verb|trans/3| and \verb|emit/3|). It
is often convenient, as in this case, to use CLP(R) constraints (by
including the directive \verb|:- use_module(library(clpr)).|) as 
in this way the expressions can be used in multiple directions and 
the same clauses can be used both to sample and to evaluate the weight the sample on the basis
of evidence,
otherwise different clauses have to be written.
In case random variables are not sufficiently instantiated to 
exploit expressions for inferring the values of other variables, 
inference will return an error.

\subsubsection{Distributional Clauses Syntax}
\label{dc}
You can also use the syntax of Distributional Clauses (DC) \cite{Nitti2016}.
Continuous random variables are represented in this case by term whose distribution can be specified with density atoms as in
\begin{verbatim}
T~Density' := Body.
\end{verbatim}
Here \verb|:=| replaces the implication symbol, \verb|T| is a term and \verb|Density'| is one of the density atoms above witthout the \verb|Var| argument, because \verb|T|
itself represents a random variables. In the body of clauses you can use the infix operator \verb|~=| to equate a term representing a random variable with a logical variable or
a constant as in \verb|T ~= X|. Internally \verb|cplint| transforms the terms representing random variables into atoms with an extra argument for holding the variable.
DC can be used to represent also discrete distributions using
\begin{verbatim}
T~uniform(L) := Body.
T~finite(D) := Body.
\end{verbatim} 
where \verb|L| is a list of values and \verb|D| is a list of couples \verb|P:V| with \verb|P| a probability and \verb|V| a value.
If \verb|Body| is empty, as in regular Prolog, the implication symbol \verb|:=| can be omitted.
\section{Semantics}
\label{semantics}

The semantics of LPADs for the case of programs without functions symbols can be given
as follows. An LPAD defines a probability distribution over normal logic programs called
\emph{worlds}. A world is obtained from an LPAD by first grounding it and, by 
selecting a single head atom for each ground clause and by including in the world
the clause with the selected head atom and the body.
The probability of a world is the product of the probabilities associated to the 
heads selected.
The probability of a ground atom (the query) is given by the sum of the probabilities
of the worlds where the query is true.

If the LPAD contains function symbols, the definition is more complex, see
\cite{DBLP:journals/ai/Poole97,DBLP:journals/jair/SatoK01,Rig15-PLP-IW}.
\section{Inference}
\label{inf}
\texttt{cplint} answers queries using the module \verb|pita|. It performs the program transformation technique of \cite{RigSwi10-ICLP10-IC}. Differently from that work, techniques alternative to tabling and answer subsumption are used.

For answering queries, you have to prepare a Prolog file where you first load \texttt{pita} and then enclose the probabilistic 
clauses in \texttt{:-cplint.} and \texttt{:-end\_cplint.} For example, the coin program above can be stored in \href{http://cplint.lamping.unife.it/example/coin.pl}{\texttt{coin.pl}} as follows
\begin{verbatim}
:- use_module(library(pita)).
:- cplint.

heads(Coin):1/2 ; tails(Coin):1/2:- 
toss(Coin),\+biased(Coin).

heads(Coin):0.6 ; tails(Coin):0.4:- 
toss(Coin),biased(Coin).

fair(Coin):0.9 ; biased(Coin):0.1.

toss(coin).

:- end_cplint.
\end{verbatim}


\verb|cplint| on SWISH allows two types of programs: LPAD and Prolog. In the first, you create a new LPAD in the editor, in the latter a Prolog program. 

In LPADs, you write in the editor the program without the library import and the compiler directives. \verb|pita| is used for inference.

You ask queries by writing the atom of which you want to compute the probability. In the coin example, the probability of \verb|heads(coin)| can be obtained with the 
\begin{verbatim}
?- heads(coin).
\end{verbatim}

In Prolog programs, you have to enter the program above (\verb|coin.pl|).

\subsection{Unconditional Queries}
\label{uncondq}
You can ask the unconditional probability of an atom using \verb|pita| with the predicate
\begin{verbatim}
prob(:Query:atom,-Probability:float).
\end{verbatim}
as in
\begin{verbatim}
?- prob(heads(coin),P).
\end{verbatim}
If the query is non-ground, \verb|prob/2| returns in backtracking the succesful instantiations together with their probability.

When using \verb|mcintyre|, the predicate for querying is 
\begin{verbatim}
mc_prob(:Query:atom,-Probability:float).
\end{verbatim} as in
\begin{verbatim}
?- mc_prob(heads(coin),P).
\end{verbatim}
With \verb|mcintyre|, you can also take a given number of sample with
\begin{verbatim}
mc_sample(:Query:atom,+Samples:int,-Successes:int,-Failures:int,
  -Probability:float).
\end{verbatim}
as in (\href{http://cplint.lamping.unife.it/example/inference/coinmc.pl}{\texttt{coinmc.pl}})
\begin{verbatim}
?- mc_sample(heads(coin),1000,S,F,P).
\end{verbatim}
that samples \verb|heads(coin)| 1000 times and returns in \verb|S| the number of successes, in \verb|F| the number of failures and in \verb|P| the
estimated probability (\verb|S/1000|).

Differently from exact inference, in approximate inference the query can be a conjunction of atoms.

If you are just interested in the probability, you can use
\begin{verbatim}
mc_sample(:Query:atom,+Samples:int,-Probability:float) 
\end{verbatim}
as in (\href{http://cplint.lamping.unife.it/example/inference/coinmc.pl}{\texttt{coinmc.pl}})
\begin{verbatim}
?- mc_sample(heads(coin),1000,Prob).
\end{verbatim}
that samples \verb|heads(coin)| 1000 times and returns the
estimated probability that a sample is true (i.e., that a sample succeeds).


Moreover, you can sample arguments of queries with
\begin{verbatim}
mc_sample_arg(:Query:atom,+Samples:int,?Arg:var,-Values:list).
\end{verbatim}
The predicate samples \verb|Query| a number of \verb|Samples| times. 
\verb|Arg| should be a variable in \verb|Query|.
The predicate returns in \verb|Values| a list of couples \verb|L-N| where
\verb|L| is the list of values of \verb|Arg| for which \verb|Query|
succeeds in a world sampled at random and \verb|N|
is the number of samples returning that list of values.
If \verb|L| is the empty list, it means that for that
sample the query failed. 
If \verb|L| is a list with a 
single element, it means that for that sample the query is 
determinate. 
If, in all couples \verb|L-N|, \verb|L| 
is a list with a 
single element, it means that the clauses in the program 
are mutually exclusive, i.e., that in every sample, only
one clause for each subgoal has the body true. This is one
of the assumptions taken for programs of the PRISM system \cite{DBLP:journals/jair/SatoK01}.
For example
\href{http://cplint.lamping.unife.it/example/inference/pfcglr.pl}{\texttt{pfcglr.pl}} and \href{http://cplint.lamping.unife.it/example/inference/plcg.pl}{\texttt{plcg.pl}} satisfy this constraint while
 \href{http://cplint.lamping.unife.it/example/inference/markov_chain.pl}{\texttt{markov\_chain.pl}} and \href{http://cplint.lamping.unife.it/example/inference/var_obj.pl}{\texttt{var\_obj.pl}} don't.


An example of use of the above predicate is
\begin{verbatim}
?- mc_sample_arg(reach(s0,0,S),50,S,Values). 
\end{verbatim}
of \href{http://cplint.lamping.unife.it/example/inference/markov_chain.pl}{\texttt{markov\_chain.pl}}
that takes 50 samples of \verb|L| in \verb|findall(S,(reach(s0,0,S),L)|.

You can sample arguments of queries also with
\begin{verbatim}
mc_sample_arg_first(:Query:atom,+Samples:int,?Arg:var,-Values:list).
\end{verbatim}
samples \verb|Query| a number of \verb|Samples| times 
and returns in \verb|Values| a list of couples \verb|V-N| where 
\verb|V| is the value of \verb|Arg| returned as the first answer by \verb|Query| in 
a world sampled at random and \verb|N| is the number of samples
returning that value.
\verb|V| is failure if the query fails.
\verb|mc_sample_arg_first/4| differs from \verb|mc_sample_arg/4| because the first just computes the first
answer of the query for each sampled world.

%Alternatively, you can use
%\begin{verbatim}
%mc_sample_arg_one(:Query:atom,+Samples:int,?Arg:var,-Values:list)
%\end{verbatim}
%that samples \verb|Query| a number of \verb|Samples| times 
%and returns in \verb|Values| a list of couples \verb|V-N| where 
%\verb|V| is a value sampled with uniform probability from those returned 
%by \verb|Query| in a world sampled at random and \verb|N| is the number of samples
%returning that value.
%\verb|V| is failure if the query fails.

Finally, you can compute expectations with 
\begin{verbatim}
mc_expectation(:Query:atom,+N:int,?Arg:var,-Exp:float).
\end{verbatim}
that computes the expected value of \verb|Arg| in \verb|Query| by
sampling.
It takes \verb|N| samples of \verb|Query| and sums up the value of \verb|Arg| for
each sample. The overall sum is divided by \verb|N| to give \verb|Exp|.

An example of use of the above predicate is
\begin{verbatim}
?- mc_expectation(eventually(elect,T),1000,T,E).
\end{verbatim}
of \href{http://cplint.lamping.unife.it/example/inference/pctl_slep.pl}{\texttt{pctl\_slep.pl}}
that returns in \verb|E| the expected value of \verb|T| by taking 1000 samples.

\subsection{Conditional Queries}
\label{condq}
You can ask the conditional probability of an atom given another atom using \verb|pita| with the predicate 
\begin{verbatim}
prob(:Query:atom,:Evidence:atom,-Probability:float).
\end{verbatim}
as in
\begin{verbatim}
?- prob(heads(coin),biased(coin),P).
\end{verbatim}
If the query/evidence are non-ground, \verb|prob/3| returns in backtracking ground instantiations together with their probability.

If the evidence is composed of more than one atom, add a clause of the form
\begin{verbatim}
evidence:- e1,...,en.
\end{verbatim}
to the program, where \verb|e1,...,en| are the evidence atoms, and use the query
\begin{verbatim}
?- prob(goal,evidence,P).
\end{verbatim}


When using \verb|mcintyre|, you can ask conditional queries with rejection sampling or with Metropolis-Hastings Markov Chain Monte Carlo.
In rejection sampling, you first query the evidence and, if the query is successful, query the goal in the same sample, otherwise
the sample is discarded.
In Metropolis-Hastings MCMC, \verb|mcintyre| follows the algorithm proposed in \cite{nampally2014adaptive} (the non adaptive version):
after a sample,  a sampled probabilistic choice is deleted and the others are retained for the next sample.
The sample is accepted with a probability of $\min\{1,\frac{N_0}{N_1}\}$ where $N_0$ is the number of choices sampled
in the previous sample and $N_1$ is the number of choices sampled in the current sample.
Since the proof in \cite{nampally2014adaptive} that the above acceptance
probability yields a valid
Metropolis-Hastings algorithm holds also when forgetting more than one 
sampled probabilistic choice, a user defined number of sampled probabilistic choices are deleted (parameter\verb|Lag|).


You can take a given number of sample with rejection sampling using
\begin{verbatim}
mc_rejection_sample(:Query:atom,:Evidence:atom,+Samples:int,
  -Successes:int,-Failures:int,-Probability:float).
\end{verbatim}
as in (\href{http://cplint.lamping.unife.it/example/inference/coinmc.pl}{\texttt{coinmc.pl}})
\begin{verbatim}
?- mc_rejection_sample(heads(coin),biased(coin),1000,S,F,P).
\end{verbatim}
that takes 1000 samples where \verb|biased(coin)| is true and returns in \verb|S| the number of samples where 
\verb|heads(coin)| is true, in \verb|F| the number of samples where \verb|heads(coin)| is false and in \verb|P| the
estimated probability (\verb|S/1000|).

Differently from exact inference, in approximate inference the evidence can be a conjunction of atoms.

You can take a given number of sample with Metropolis-Hastings MCMC using
\begin{verbatim}
mc_mh_sample(:Query:atom,:Evidence:atom,+Samples:int,+Lag:int,
  -Successes:int,-Failures:int,-Probability:float).
\end{verbatim}
where \verb|Lag| is the number of sampled choices to forget before taking a new sample.
For example (\href{http://cplint.lamping.unife.it/example/inference/arithm.pl}{\texttt{arithm.pl}})
\begin{verbatim}
?- mc_mh_sample(eval(2,4),eval(1,3),10000,1,T,F,P).
\end{verbatim}
takes 10000 accepted samples and returns in \verb|T| the number of samples where 
\verb|eval(2,4)| is true, in \verb|F| the number of samples where \verb|eval(2,4)| is false and in \verb|P| the
estimated probability (\verb|T/10000|).


Moreover, you can sample arguments of queries with rejection sampling and Metropolis-Hastings MCMC using
\begin{verbatim}
mc_rejection_sample_arg(:Query:atom,:Evidence:atom,
  +Samples:int,?Arg:var,-Values:list).
mc_mh_sample_arg(:Query:atom,:Evidence:atom,
  +Samples:int,+Lag:int,?Arg:var,-Values:list).
\end{verbatim}
that return the distribution of values for \verb|Arg| in \verb|Query| in \verb|Samples| of
\verb|Query| given that \verb|Evidence| is true.
The predicate returns in \verb|Values| a list of couples \verb|L-N| where
\verb|L| is the list of values of \verb|Arg| for which \verb|Query|
succeeds in a world sampled at random where \verb|Evidence| is true and \verb|N|
is the number of samples returning that list of values.

An example of use of the above predicates is
\begin{verbatim}
?- mc_mh_sample_arg(eval(2,Y),eval(1,3),1000,1,Y,V).
\end{verbatim}
of \href{http://cplint.lamping.unife.it/example/inference/arithm.pl}{\texttt{arithm.pl}}.

Finally, you can compute expectations with 
\begin{verbatim}
mc_expectation(:Query:atom,+N:int,?Arg:var,-Exp:float).
\end{verbatim}
that computes the expected value of \verb|Arg| in \verb|Query| by
sampling.
It takes \verb|N| samples of \verb|Query| and sums up the value of \verb|Arg| for
each sample. The overall sum is divided by \verb|N| to give \verb|Exp|.

An example of use of the above predicate is
\begin{verbatim}
?- mc_expectation(eventually(elect,T),1000,T,E).
\end{verbatim}
of \href{http://cplint.lamping.unife.it/example/inference/pctl_slep.pl}{\texttt{pctl\_slep.pl}}
that returns in \verb|E| the expected value of \verb|T| by taking 1000 samples.

To compute conditional expectations, use
\begin{verbatim}
mc_mh_expectation(:Query:atom,:Evidence:atom,+N:int,
  +Lag:int,?Arg:var,-Exp:float).
\end{verbatim}
as in
\begin{verbatim}
?- mc_mh_expectation(eval(2,Y),eval(1,3),1000,1,Y,E).
\end{verbatim}
of \href{http://cplint.lamping.unife.it/example/inference/arithm.pl}{\texttt{arithm.pl}}
that computes the expectation of argument \verb|Y| of \verb|eval(2,Y)| given that 
\verb|eval(1,3)| is true by taking 1000 samples using Metropolis-Hastings MCMC.

When you have continuous random variables, you may be interested in 
sampling arguments of goals representing continuous random variables.
In this way you can build a probability density of the sampled argument.
When you do not have evidence or you have evidence on atoms not depending
on continuous random variables, you can use the above predicates for sampling
arguments.

For example
\begin{verbatim}
?- mc_sample_arg(value(0,X),1000,X,L).
\end{verbatim}
from (\href{http://cplint.lamping.unife.it/example/inference/gauss_mean_est.pl}{\texttt{gauss\_mean\_est.pl}})) samples 1000 values for \verb|X| in
\verb|value(0,X)| and returns them in \verb|L|.

When you have evidence on ground atoms that have continuous values as 
arguments, you cannot use rejection sampling or Metropolis-Hastings,
as the probability of the evidence is 0. 
Instead, you can use likelihood weighting \cite{fung1990weighing,koller2009probabilistic} to obtain samples of 
continuous arguments of a goal. The predicate 
\begin{verbatim}
mc_lw_sample_arg(:Query:atom,:Evidence:atom,+N:int,?Arg:var,-ValList)
\end{verbatim}
returns in \verb|ValList| a list of \verb|N| values of argument
\verb|Arg| of goal \verb|Query| given \verb|Evidence| (a conjunction of atoms is allowed here).
For example
\begin{verbatim}
?- mc_lw_sample_arg(value(0,X),(value(1,9),value(2,8)),100,X,L).
\end{verbatim}
from (\href{http://cplint.lamping.unife.it/example/inference/gauss_mean_est.pl}{\texttt{gauss\_mean\_est.pl}})) samples 100 values for \verb|X| in
\verb|value(0,X)| given that \verb|value(1,9)| and \verb|value(2,8)| have been observed.



You can also see the probability of the query being true and 
being false as a bar chart with \verb|prob_bar(:Query:atom,-Probability:dict)| as in
\begin{verbatim}
?- prob_bar(heads(coin),P).
\end{verbatim}
if you include
\begin{verbatim}
:- use_rendering(c3).
\end{verbatim}
before \verb|:- pita.| \verb|P| will be instantiated with a
dict for rendering with \verb|c3|. It will be shown as a bar chart with
a bar for the probability of \verb|heads(coin)| true and a bar for the probability of \verb|heads(coin)| false.


When using \verb|mcintyre|, you can use
\begin{verbatim}
mc_prob_bar(:Query:atom,-Probability:dict).
\end{verbatim}
as in
\begin{verbatim}
?- mc_prob_bar(heads(coin),P).
\end{verbatim}
to obtain a chart representation of the probability.

You can obtain a bar chart of the samples with
\begin{verbatim}
?- mc_sample_bar(heads(coin),1000,Chart).
\end{verbatim}
that returns in \verb|Chart| a diagram with one bar for the number of successes and 
one bar for the number of failures.

You can also graph the results of sampling arguments with
\begin{verbatim}
mc_sample_arg_bar(:Query:atom,+Samples:int,?Arg:var,-Chart:dict).
mc_sample_arg_first_bar(:Query:atom,+Samples:int,
  ?Arg:var,-Chart:dict).
mc_rejection_sample_arg_bar(:Query:atom,:Evidence:atom,+Samples:int,
  ?Arg:var,-Chart:dict).
mc_mh_sample_arg_bar(:Query:atom,:Evidence:atom,+Samples:int,
  +Lag:int,?Arg:var,-Chart:dict).
\end{verbatim} 
that return in \verb|Chart| a bar chart with a bar for each possible sampled value whose size is the number of samples
returning that value.

An example is
\begin{verbatim}
?- mc_sample_arg_bar(reach(s0,0,S),50,S,Chart). 
\end{verbatim}
of \href{http://cplint.lamping.unife.it/example/inference/markov_chain.pl}{\texttt{markov\_chain.pl}}.

\subsection{Parameters}
The module makes use of a number of parameters in order to control its behavior. They can be set with the directive
\begin{verbatim}
:- set_pita(<parameter>,<value>).
\end{verbatim}
inside the couple \texttt{:-cplint.} and \texttt{:-end\_cplint.} 

The current value can be read with
\begin{verbatim}
?- setting_pita(<parameter>,Value).
\end{verbatim}
from the top-level.
The available parameters are:
\begin{itemize}
\item 
	 \verb|epsilon_parsing|: if (1 - the sum of the probabilities of all the head atoms) is larger than 
    \verb|epsilon_parsing|,
		then \texttt{pita} adds the null event to the head. Default value \texttt{0.00001}.
\item \verb|single_var|: determines how non ground clauses are treated: if \texttt{true}, a single random variable is assigned to the whole non ground clause, 
if \texttt{false}, a different random variable is assigned to every grounding of the clause. Default value \texttt{false}.
\item \verb|depth_bound|: if \texttt{true}, the depth of the derivation of the goal is limited to the value of the \texttt{depth} parameter.  Default value \texttt{false}.
\item  \texttt{depth}: maximum depth of derivations when  \verb|depth_bound| is set to \texttt{true}. Default value \texttt{2}.
\end{itemize}



\section{Learning}
\label{learning}
The following learning algorithms are available:
\begin{itemize}
\item EMBLEM (EM over Bdds for probabilistic Logic programs Efficient Mining): an implementation of EM for learning parameters that computes expectations directly on BDDs \cite{BelRig11-IDA}, \cite{BelRig11-CILC11-NC}, \cite{BelRig11-TR}
\item SLIPCOVER (Structure LearnIng of Probabilistic logic programs by searChing OVER the clause space): an algorithm for learning the structure of programs by searching the clause space and the theory space separately \cite{BelRig13-TPLP-IJ}
\end{itemize}

\subsection{Input}
To execute the learning algorithms, prepare a Prolog file divided in five parts
\begin{itemize}
\item preamble
\item  background knowledge, i.e., knowledge valid for all interpretations
\item  LPAD/CPL-program for you which you want to learn the parameters (optional)
\item language bias information
\item  example interpretations 
\end{itemize}
The preamble must come first, the order of the other parts can be changed.

For example, consider the Bongard problems of \cite{RaeLae95-ALT95}. 
The \texttt{pack/cplint/ prolog/examples/learning} folder in SWI-Prolog home contains some example learning files. 
For example, it contains \verb|bongard.pl| and \verb|bongardkeys.pl| that represent a Bongard problem.
Let us consider \verb|bongard.pl|.

\subsubsection{Preamble}
In the preamble, the SLIPCOVER library is loaded with
\begin{verbatim}
:- use_module(library(slipcover)).
\end{verbatim}
Then, if you are using your file in cplint on SWISH, you could add
\begin{verbatim}
:- if(current_predicate(use_rendering/1)).
:- use_rendering(c3).
:- use_rendering(lpad).
:- endif.
\end{verbatim}
if you want a nice representation of the output (in particular, if you want graphs of the ROC and PR curves).

Now you can initialize SLIPCOVER with
\begin{verbatim}
:- sc.
\end{verbatim}
At this point you can start setting parameters for SLIPCOVER such as for example
\begin{verbatim}
:- set_sc(megaex_bottom,20).
:- set_sc(max_iter,2).
:- set_sc(max_iter_structure,5).
:- set_sc(verbosity,1).
\end{verbatim}
We will see later the list of available parameters.
This ends the preamble.

\subsubsection{Backgroung and Initial LPAD/CPL-program}
%
Now you can specify the background knowledge with a 
fact of the form 
\begin{verbatim}
bg(<list of terms representing clauses>).
\end{verbatim}
and an initial program with a fact of the form 
\begin{verbatim}
in(<list of terms representing clauses>).
\end{verbatim}
The initial program is used in parameter learning for providing 
the structure. The indicated parameters do not matter as they are first randomized.
Remember to enclose each clause in parentheses because \verb|:-| has the highest precedence.

For example, \verb|bongard.pl| has the initial program 
\begin{verbatim}
in([(pos:0.197575 :-
       circle(A),
       in(B,A)),
    (pos:0.000303421 :-
       circle(A),
       triangle(B)), 
    (pos:0.000448807 :-
       triangle(A),
       circle(B))]).
\end{verbatim}
Both facts should be present. If there are no background/input clauses then write \verb|bg([]).|/\verb|in([]).|


\subsubsection{Language Bias}
%
The language bias part contains the declarations of the input and output predicates.
Output predicates are declared as
\begin{verbatim}
output(<predicate>/<arity>).
\end{verbatim}
and indicate the predicate whose atoms you want to predict. Derivations for the atoms for this predicates in the input data
are built by the systems.

Input predicates are those whose atoms you are not interested in predicting. You can declare closed world input predicates with
\begin{verbatim}
input_cw(<predicate>/<arity>).
\end{verbatim}
For these predicates, the only true atoms are those in the interpretations, the clauses in the input/hypothesized program are not used to derive atoms not present in the interpretations.

Open world input predicates are declared with
\begin{verbatim}
input(<predicate>/<arity>).
\end{verbatim}
In this case, if a subgoal for such a predicate is encountered when deriving a subgoal for the output predicates, 
both the facts in the interpretations and the clauses of the input program are used.

Then, you have to specify the language bias by means of mode declarations in the style of 
\href{http://www.doc.ic.ac.uk/\string ~shm/progol.html}{Progol}.
\begin{verbatim}
modeh(<recall>,<predicate>(<arg1>,...).
\end{verbatim}
specifies the atoms that can appear in the head of clauses, while
\begin{verbatim}
modeb(<recall>,<predicate>(<arg1>,...).
\end{verbatim}
specifies the atoms that can appear in the body of clauses.
\texttt{<recall>} can be an integer or \texttt{*}.
\texttt{<recall>} indicates how many atoms for the predicate specification are
retained in the bottom clause during a saturation step. \texttt{*} stands for all those that are found. Otherwise the indicated number is randomly chosen.

Arguments of the form
\begin{verbatim}
+<type>
\end{verbatim}
specifies that the argument should be an input variable of type \texttt{<type>}, i.e., a variable replacing a \texttt{+<type>} argument in the head or a \texttt{-<type>} argument in a preceding literal in the current hypothesized clause.

Another argument form is
\begin{verbatim}
-<type>
\end{verbatim}
for specifying that the argument should be a output variable of type \texttt{<type>}. 
Any variable can replace this argument, either input or output.
The only constraint on output variables is that those in the head of the current hypothesized 
clause must appear as output variables in an atom of the body.

Other forms are
\begin{verbatim}
#<type>
\end{verbatim}
for specifying an argument which should be replaced by a constant of type \texttt{<type>} in the bottom clause but should not be used for replacing input variables of the following literals when building the bottom clause or 
\begin{verbatim}
-#<type>
\end{verbatim}
for specifying an argument which should be replaced by a constant of type \texttt{<type>} in the bottom clause and that should be used for replacing input variables of the following literals when building the bottom clause. 
%\verb|#| and \verb|-#| differ only in the creation of the bottom clause.
\begin{verbatim}
<constant>
\end{verbatim}
for specifying a constant.

An example of language bias for the Bongard domain is
\begin{verbatim}
output(pos/0).

input_cw(triangle/1).
input_cw(square/1).
input_cw(circle/1).
input_cw(in/2).
input_cw(config/2).

modeh(*,pos).
modeb(*,triangle(-obj)).
modeb(*,square(-obj)).
modeb(*,circle(-obj)).
modeb(*,in(+obj,-obj)).
modeb(*,in(-obj,+obj)).
modeb(*,config(+obj,-#dir)).
\end{verbatim}
SLIPCOVER also requires facts for the \verb|determination/2| predicate  Aleph-style that indicate which predicates can appear in the body of clauses. 
For example
\begin{verbatim}
determination(pos/0,triangle/1).
determination(pos/0,square/1).
determination(pos/0,circle/1).
determination(pos/0,in/2).
determination(pos/0,config/2).
\end{verbatim}
state that \verb|triangle/1| can appear in the body of clauses for \verb|pos/0|.

SLIPCOVER also allows mode declarations of the form
\begin{verbatim}
modeh(<r>,[<s1>,...,<sn>],[<a1>,...,<an>],[<P1/Ar1>,...,<Pk/Ark>]). 
\end{verbatim}
These mode declarations are used to generate clauses with more than two head atoms. In them, \verb|<s1>,...,<sn>| are schemas,  \verb|<a1>,...,<an>| are atoms such that \verb|<ai>| is obtained from $\verb|<si>|$ by replacing placemarkers with variables, 
\verb|<Pi/Ari>| are the predicates admitted in the body. \verb|<a1>,...,<an>| are used to indicate which variables should be shared by the atoms in the head.
An example of such a mode declaration (from \verb|uwcselearn.pl|) is
\begin{verbatim}
modeh(*,
[advisedby(+person,+person),tempadvisedby(+person,+person)],
[advisedby(A,B),tempadvisedby(A,B)],
[professor/1,student/1,hasposition/2,inphase/2,
publication/2,taughtby/3,ta/3,courselevel/2,yearsinprogram/2]).
\end{verbatim}

\subsubsection{Example Interpretations}
The last part of the file contains the data.
You can specify data with two modalities:
models and keys.
In the models type, you specify an example model (or interpretation or megaexample) as a list of Prolog facts initiated by 
\texttt{begin(model(<name>)).} and terminated by \texttt{end(model(<name>)).} as in
\begin{verbatim}
begin(model(2)).
pos.
triangle(o5).
config(o5,up).
square(o4).
in(o4,o5).
circle(o3).
triangle(o2).
config(o2,up).
in(o2,o3).
triangle(o1).
config(o1,up).
end(model(2)).
\end{verbatim}
The interpretations may contain a fact of the form
\begin{verbatim}
prob(0.3).
\end{verbatim}
assigning a probability (0.3 in this case) to the interpretations. If this is omitted, the probability of each interpretation is considered equal to $1/n$ where $n$ is the total number of interpretations. \verb|prob/1| can be used to set a different multiplicity for the interpretations.

The facts in the interpretation are loaded in SWI-Prolog database by adding an extra initial argument equal to the name of the model.

Alternatively, with the keys modality, you can directly write the facts and the first argument will be interpreted as a model identifier. The above interpretation in the keys modality is
\begin{verbatim}
pos(2).
triangle(2,o5).
config(2,o5,up).
square(2,o4).
in(2,o4,o5).
circle(2,o3).
triangle(2,o2).
config(2,o2,up).
in(2,o2,o3).
triangle(2,o1).
config(2,o1,up).
\end{verbatim}
which is contained in the \verb|bongardkeys.pl|
This is also how model \verb|2| above is stored in SWI-Prolog database.
The two modalities, models and keys, can be mixed in the same file.

Note that you can add background knowledge that is not probabilistic directly to the file writing the clauses taking into account the model argument. For example \verb|carc.pl|
contains

\begin{verbatim}
connected(_M,Ring1,Ring2):-
  Ring1 \= Ring2,
  member(A,Ring1),
  member(A,Ring2), !.

symbond(Mod,A,B,T):- bond(Mod,A,B,T).
symbond(Mod,A,B,T):- bond(Mod,B,A,T).
\end{verbatim}
where the first argument of all the atoms is the model.

Then you must indicate how the examples are divided in folds with facts of the form:
\verb|fold(<fold_name>,<list of model identifiers>)|, as for example
\begin{verbatim}
fold(train,[2,3,...]).
fold(test,[490,491,...]).
\end{verbatim}


\subsection{Commands}
\subsubsection{Parameter Learning}
To execute EMBLEM, prepare an input file in the editor panel as indicated above 
and call
\begin{verbatim}
?- induce_par(<list of folds>,P).
\end{verbatim}
where \verb|<list of folds>| is a list of the folds for training and
\verb|P| will contain the input program with updated parameters.

For example \href{http://cplint.lamping.unife.it/example/bongard.pl}{\texttt{bongard.pl}}, you can 
perform parameter learning on the \verb|train| fold with 
\begin{verbatim}
?- induce_par([train],P).
\end{verbatim}
A program can also be tested on a test set with
\begin{verbatim}
?- test(<program>,<list of folds>,LL,AUCROC,ROC,AUCPR,PR).
\end{verbatim}
where \verb|<program>| is a list of terms representing clauses and
\verb|<list of folds>| is a list of folds.
This returns the log likelihood of the test examples in \verb|LL|, the Area Under the ROC curve in \verb|AUCROC|, a dictionary containing the list of points (in the form of Prolog pairs \verb|x-y|) of the ROC curve in \verb|ROC|,
the Area Under the PR curve in \verb|AUCPR|, a dictionary containing the list of points of the PR curve in \verb|PR|.

For example, to test on fold \verb|test| the program learned on fold \verb|train| you can run the query
\begin{verbatim}
?- induce_par([train],P),
   test(P,[test],LL,AUCROC,ROC,AUCPR,PR).
\end{verbatim}
Or you can test the input program on the fold \verb|test| with
\begin{verbatim}
?- in(P),
   test(P,[test],LL,AUCROC,ROC,AUCPR,PR).
\end{verbatim}
By including
\begin{verbatim}
:- use_rendering(c3).
:- use_rendering(lpad).
\end{verbatim}
in the code before \verb|:- sc.| the curves will be shown as graphs and the output program will be pretty printed.



\subsubsection{Structure Learning}
To execute SLIPCOVER,
prepare an input file in the editor panel as indicated above 
and call
\begin{verbatim}
?- induce(<list of folds>,P).
\end{verbatim}
where \verb|<list of folds>| is a list of the folds for training and
\verb|P| will contain the learned program.

For example \href{http://cplint.lamping.unife.it/example/bongard.pl}{\texttt{bongard.pl}}, you can perform structure learning on the \verb|train| fold with 
\begin{verbatim}
?- induce([train],P).
\end{verbatim}
A program can also be tested on a test set with \verb|test/7| as
described above.

\subsection{Parameters}
Parameters are set with  commands of the form
\begin{verbatim}
:- set_sc(<parameter>,<value>).
\end{verbatim}
The available parameters are:
\begin{itemize}
\item \verb|specialization|: (values: \verb|{bottom,mode}|,  default value: \texttt{bottom}, valid for SLIPCOVER) specialization mode. 
\item \verb|depth_bound|: (values: \verb|{true,false}|,  default value: \texttt{true}) if \texttt{true}, the depth of the derivation of the goal is limited to the value of the \texttt{depth} parameter. 
\item \verb|depth| (values: integer, default value: 2): depth of derivations if  \verb|depth_bound|  is set to \verb|true|
\item \verb|single_var| (values: \verb|{true,false}|, default value: \verb|false|): if set to \verb|true|, there is a random variable for each clause, instead of a different random variable for each grounding of each clause
\item \verb|epsilon_em| (values: real, default value: 0.1): if the difference in the log likelihood in two successive parameter EM iteration is smaller
than \verb|epsilon_em|, then EM stops 
\item \verb|epsilon_em_fraction| (values: real, default value: 0.01): if the difference in the log likelihood in two successive parameter EM iteration is smaller
than \verb|epsilon_em_fraction|*(-current log likelihood), then EM stops
\item \verb|iter| (values: integer, defualt value: 1): maximum number of iteration of EM parameter learning. If set to -1, no maximum number of iterations is imposed
\item \verb|iterREF| (values: integer, defualt value: 1, valid for  
 SLIPCOVER and LEMUR):
 maximum number of iteration of EM parameter learning for refinements. If set to -1, no maximum number of iterations is imposed.
\item \verb|random_restarts_number| (values: integer, default value: 1, valid for EMBLEM, SLIPCOVER and LEMUR): number of random restarts of parameter EM learning
\item \verb|random_restarts_REFnumber| (values: integer, default value: 1, valid for  SLIPCOVER and LEMUR): number of random restarts of parameter EM learning for refinements
\item \verb|seed| (values: seed(integer) or seed(random), default value \verb|seed(3032)|): seed for the Prolog random functions, see \href{http://www.swi-prolog.org/pldoc/man?predicate=set_random/1}{SWI-Prolog manual}
\item \verb|c_seed| (values: unsigned integer, default value 21344)): seed for the C random functions
\item \verb|logzero| (values: negative real, default value $\log(0.000001)$): value assigned to $\log 0$
\item \verb|max_iter| (values: integer, default value: 10, valid for  SLIPCOVER): number of interations of beam search
\item \verb|max_var| (values: integer, default value: 4, valid for 
SLIPCOVER and LEMUR): maximum number of distinct variables in a clause
\item \verb|beamsize|  (values: integer, default value: 100, valid for SLIPCOVER): size of the beam 
\item \verb|megaex_bottom| (values: integer, default value: 1, valid for SLIPCOVER): number of mega-examples on which to build the bottom clauses
\item \verb|initial_clauses_per_megaex| (values: integer, default value: 1, valid for SLIPCOVER): 
 number of bottom clauses to build for each mega-example (or 
 model or interpretation)
\item \verb|d| (values: integer, default value: 1, valid for SLIPCOVER): 
 number of saturation steps when building the bottom clause
\item \verb|mcts_beamsize|  (values: integer, default value: 3, valid for LEMUR): size of the Monte-Carlo tree search beam
\item \verb|mcts_visits|  (values: integer, default value: +1e20, valid for LEMUR): maximum number of visits 
\item \verb|max_iter_structure| (values: integer, default value: 10000, valid for SLIPCOVER): 
maximum  number of theory search iterations
\item \verb|background_clauses| (values: integer, default value: 50, valid for SLIPCOVER): 
 maximum numbers of background clauses
\item \verb|maxdepth_var| (values: integer, default value: 2, valid for SLIPCOVER and LEMUR): maximum depth of
variables in clauses (as defined in \cite{DBLP:journals/ai/Cohen95}).
\item \verb|mcts_max_depth|  (values: integer, default value: 8, valid for LEMUR): maximum depth of default policy search
\item \verb|mcts_c|  (values: real, default value: 0.7, valid for LEMUR): value of parameter $C$ in the computation of UCT
\item \verb|mcts_iter|  (values: integer, default value: 20, valid for LEMUR): number of Monte-Carlo tree search iterations
\item \verb|mcts_maxrestarts|  (values: integer, default value: 20, valid for LEMUR): maximum number of Monte-Carlo tree search restarts
\item \verb|neg_ex| (values:  \verb|given|, \verb|cw|, default value: \verb|cw|): if  set to \verb|given|, the negative examples in testing
are taken from the test folds interpretations, i.e., those examples \verb|ex| stored as \verb|neg(ex)|; if set to \verb|cw|, the negative examples are generated according to the closed world assumption, i.e., all atoms for target predicates that are not positive examples. The set of all atoms is obtained by collecting the set of constants for each type of the arguments of the target predicate.
\item \verb|alpha| (values: floating point $\geq 0$, default value: 0): parameter of the 
symmetric Dirichlet distribution used to initialize the parameters. If it takes value 0, a truncated Dirichlet process is
used to sample parameters: the probability of being true of  each Boolean random variable 
used to represent multivalued
random variables is sampled uniformly and independently in [0,1]. If it takes a value $\geq 0$, the parameters are sampled from
a symmetric Dirichlet distribution, i.e. a Dirichlet distribution  with vector of parameters $(\alpha,\ldots,\alpha)$.
\item \verb|verbosity| (values: integer in [1,3], default value: 1): level of verbosity of the algorithms.
\end{itemize}


\section{Manual in PDF}
A PDF version of the manual is available at
\url{https://github.com/friguzzi/cplint/blob/master/doc/help-cplint.pdf}.
\section{Bibliography}
\bibliographystyle{plain}
\bibliography{bib}
\end{document}